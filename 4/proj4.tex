\documentclass[a4paper, 11pt]{article}
\usepackage[utf8]{inputenc}
\usepackage[czech]{babel}
\usepackage[IL2]{fontenc}
\usepackage[left=2cm, text={17cm, 24cm}, top=3cm]{geometry}
\usepackage{hyperref}
\usepackage{times}
\usepackage{multirow}




\begin{document}

\thispagestyle{empty}

\begin{titlepage}
\begin{center}
    
    {\Huge
    \textsc{
    Vysoké učení technické v Brně\\[0.4em]}
    {\huge
    \textsc{
    Fakulta informačních technologií}}}\\
    \vspace{\stretch{0.382}}
    {\LARGE
    Typografie a publikování -- 4. projekt\\[0.3em]
    {\Huge Bibliografické citace a odkazy}
    }
    \vspace{\stretch{0.618}}
\end{center}

{\Large \today \hfill
Dalibor Králik (xkrali20)}

\end{titlepage}

\section{\LaTeX}
\LaTeX\:  je typografickým nástrojom, ktorý je používaný celým svetom pre vedecké dokumenty, knihy a tak isto pre ostatné formy publikácií viz  \cite{WhyLearnLatex}. 
 \LaTeX\:  je nadstavbou pôvodného Texu, ktorý je značkovacím jazykom, umožňuje profesionálnu sadzbu dokumentov v mnoho jazykova  ich špeciálnu úpravu. \LaTeX\: ako nástroj je konkurenciou pre editori ako je napríklad MS Word alebo LibreOffice Writter. \cite{SokolMiroslav2012OLe}

\section{Ako vyzerá práca s \LaTeX om}
Systém \LaTeX\: nie je WYSIWYG editor \cite{WYSIWYG}, takže práca v ňom pripomína skôr programovanie. Práca v ňom pozostáva z troch krokov, ako je uvedené v \cite{Rybicka2003}:
    \begin{enumerate}
        \item písanie zdrojového textu,
        \item preklad,
        \item sledovanie výsledku.
    \end{enumerate}

\section{Ako sa naučiť \LaTeX}
Exituje mnoho návodov, tutoriálov a zdrojov, z ktorých sa človek môže \LaTeX \: naučiť. Avšak záleží na človeku, aká forma učenia mu vyhovuje. Pre tých, ktorým vyhovuje učenie z kníh, môžem odporučiť českú knihu \uv{\emph{LaTeX pro začátečníky}} \cite{Rybicka2003} alebo knihu v anglickom jazyku \uv{\emph{Guide to LaTeX}} \cite{KopkaHelmut2004GtL}. Pre fanúšikov učenia sa z online materiálov odporúčam stránku Overleaf \footnote{Stránka Overleaf zemaraná na systém LaTeX: \href{https://www.overleaf.com}{https://www.overleaf.com}}, ktorá poskytuje rôzne tutoriály a dokumentáciu k \LaTeX u.

\section{Výhody \LaTeX u}
Hlavnými výhodami \LaTeX u je automatické zvýrazňovamie syntaxe zdrojových kódov a farebne odlíšené komentáre. Ďalšou výhodou je profesionálna sadzba matematických vzorcov, ktoré ocenia hlavne vedecký pracovníci a autori vedeckých textov. \cite{BartlikJakub2017Sopp}

\subsection{Matematické výrazy}
Matematické výrazy sú najsilnejšiou stránkou \LaTeX u. Z mojho pohľadu sa mu nevyrovná ani matematické prostredie v MS Word. Matematická sazba v \LaTeX e sa sádza do znakov doláru \$. \TeX\: tvorí vzorčeky vo vnútornom matematickom móde \$~\dots~\$ alebo v display móde \$~\$~\dots~\$~\$~. Pravidla pre tvorbu vzorčekov sú v oboch módoch rovnaké viz \cite{Olsak2021}.

\section{Nevyhodu \LaTeX u}
Pár nevýhod \LaTeX u zhrnul Kryštof Davidek vo svojom blogu \cite{KrystofDavidek}. Medzi ne patrí napríklad inštalácia \LaTeX u, ktorá zaberá pomerne veľké množstvo úložného priestoru z dôvodu veľkého počtu balíčkov. Za hlavnú nevýhodu \LaTeX u sa považuje dĺžka a náročnosť učenia sa tohto systému. Vzhľadom na WYSIWYG editory je doba učenia násobne väčšia a náročnejšia.

\section{Sazda bibliografie podľa normy ISO 690 v systéme \LaTeX}
\begin{quote}
    Dodržiavanie normy ISO 690 pri tvorbe bibliografických odkazov a citácií býva vyžadované mnohými inštitúciami nielen v českom akademickom prostredí. V systéme LATEX však doteraz neexistovala žiadna podpora, ktorá by plnohodnotne riešila túto problematiku. Až na základe referenčnej implementácie balíka biblatex-iso690 vznikol balíček, ktorý splňuje požiadavky normy v plnom rozsahu a výrazne tak zjednodušuje citovanie informačných zdrojov. \cite{Luptak2016}
\end{quote}


\section{Konverzia \LaTeX u do XML}
Potreba preložiť \LaTeX\: do iných formátor ako je napríkald HTML alebo XML bola zistená a sledovaná už niekoľko rokov. Preto vzniklo niekoľko nástrojov na konverziu do týchto požadovaných formátov. Preto vznikli nástroje ako \texttt{latex2html} alebo \texttt{LaTeXML} viz \cite{KohlhaseMichael2008ULaa}.


\newpage
	\bibliographystyle{czechiso}
	\renewcommand{\refname}{Literatura}
	\bibliography{proj4}
\end{document}
